
If resource adaptation is triggered because of resource constraint violation and if a reduced program/component is generated, as the reduced program is still a useful artifact that will continue to function as it is mostly, we hypothesize that small local changes are applied to the program. The newly created program is a perturbation of the original program and mutation captures perturbations well. In traditional mutation testing, purpose of the mutant is to act as an artificial bug while in our case mutant is a useful perturbed version of original program. Rahul write this statement: Statement Deletion mutation operator was proposed by xyz and during empirical analysis conducted by abc it was found to be very effective and it subsumed other mutation operator significantly. When HDD is traditionally applied on programs, it relies on tree structure of the program like AST, CFG or ECFG etc. For example, picireny relied on ECFG of programs. Its reduction run even deeper then program statements, it even reduces expressions. Our approach does not go beyond SDL because (1) SDL operator was found to be subsuming other mutation operator that are used in expressions (2) As reduced program still needs to be working as mostly correct program, we expect simple removals. (3) Minimally reducing expressions, conditions etc further may be very time consuming and hence runtime adaptations may not converge quickly.

   